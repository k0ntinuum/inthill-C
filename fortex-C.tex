

\documentclass{article}
\usepackage[utf8]{inputenc}
\usepackage{setspace}
\usepackage{ mathrsfs }
\usepackage{graphicx}
\usepackage{amssymb} %maths
\usepackage{amsmath} %maths
\usepackage[margin=0.2in]{geometry}
\usepackage{graphicx}
\usepackage{ulem}
\setlength{\parindent}{0pt}
\setlength{\parskip}{10pt}
\usepackage{hyperref}
\usepackage[autostyle]{csquotes}

\usepackage{cancel}
\renewcommand{\i}{\textit}
\renewcommand{\b}{\textbf}
\newcommand{\q}{\enquote}
%\vskip1.0in





\begin{document}

{\setstretch{0.0}{


\b{Fortex} is a \q{stack of wheels}. Each disc or revolving \q{wheel of symbols} is of a different size. The leftmost symbols of all the wheels is added mod $b$ with the plaintext symbol to get the ciphertext symbol. Then the wheels are all advanced a certain number of positions or notches, depending on the value of the cipher symbol (and therefore on the plaintext.)

 

\end{document}
